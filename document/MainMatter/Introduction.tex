\chapter*{Introducción}\label{chapter:introduction}
\addcontentsline{toc}{chapter}{Introducción}
El Nodo Central de la Universidad de La Habana tiene entre sus responsabilidades dar  las credenciales  digitales a todos los usuarios de la Universidad. Los trabajadores y estudiantes de la instituci\'on registran sus datos personales en las bases de datos de recursos humanos y secretaria docente respectivamente. Esa informaci\'on  es utilizada m\'as adelante para que los usuarios puedan autenticarse en los distintos servicios de la institución, respaldado por sus sistemas de origen. 


\section*{Motivación}
Actualmente todos los usuarios de la Universidad de La Habana se almacenan en dos Protocolos Ligeros de Acceso a Directorios (en inglés: Lightweight Directory Access Protocol, también conocido por sus siglas como LDAP). En los dos directorios se guardan las cuentas de correo y los datos de sus usuarios. En uno ellos se registra la información de los estudiantes y en el otro la de los trabajadores. A partir de estos, todos los sitios web y aplicaciones de la universidad, individualmente, verifican la pertenencia del usuario a la institución.

El sistema actual es poco eficiente y requiere de intervención humana constante para corregir y/o restablecer el apropiado funcionamiento de los servicios de autenticación. Esto genera tiempos elevados de respuesta y dificulta el proceso de Transformación Digital que está siendo llevado a cabo por nuestra Universidad.

\section*{Antecedentes}
Actualmente todos los sitios web y aplicaciones de La Universidad de La Habana tienen servicios de autenticación individuales que dependen de los dos mencionados directorios que contienen a los usuarios.

\section*{Problemática}

\section*{Objetivo}
Con el propósito de presentar una propuesta para solucionar la problemática expuesta anteriormente, se plantean los siguientes objetivos:

\subsubsection*{Objetivo General}

\begin{itemize}	
	\item Diseñar e implementar un sistema de autenticación centralizada para todos los usuarios de La Universidad de La Habana. 
\end{itemize}

\subsubsection*{Objetivos Específicos}
\begin{itemize}	
	\item Generar los servicios de autenticación con compatibilidad con todas las tecnologías existentes y previstas en la institución.
	\item Implementar la gestión de control de acceso a todos los servicios ofrecidos por el Nodo Central de forma extensible a futuros servicios.
\end{itemize}