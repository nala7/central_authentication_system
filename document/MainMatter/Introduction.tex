\chapter*{Introducción}\label{chapter:introduction}
\addcontentsline{toc}{chapter}{Introducción}
El Nodo Central de la Universidad de la Habana tiene entre sus responsabilidades dar  credenciales  a todos los usuarios de la universidad. Los trabajadores y estudiantes de la instituci\'on registran sus datos en las bases de datos de recursos humanos y secretaria docente respectivamente. Esa informaci\'on luego es utilizada para automatizar la generación de cuentas en tiempo real respaldado con sus sistemas de origen.


\section*{Motivación}
Todo trabajador o estudiante al momento de ingresar en la Universidad, dígase realizar el contrato en recursos humanos u ofrecer sus datos en secretaría docente, respectivamente, debe esperar por un plazo de al menos 24 horas hasta que el sistema vigente actualiza todos sus datos, generando molestias e inquietudes en los usuarios.

El sistema actual requiere de intervención humana constantemente para corregir y/o restablecer el apropiado funcionamiento de los servicios de autenticación generando tiempos elevados de respuesta y contradiciendo de esta forma el proceso de Transformación Digital que está siendo llevado a cabo por nuestra Universidad.

\section*{Antecedentes}
El funcionamiento del sistema vigente consta de dos momentos: el primero está implementado en directorio, el cual obtiene toda su información realizando una copia parcial de los sistemas originales, borrando y recreando cada noche todos sus usuarios. Este sistema ofrece autenticación múltiple con tecnologías obsoletas por lo cual todos los navegadores clientes en la actualidad refutan su conexión a no ser que una excepción manual sea generada, todo esto genera brechas de seguridad no permisibles en la actualidad. Por otra parte, el segundo momento se encuentra en el uso de LDAP, este sistema es dependiente directamente del anterior y por tanto se considera una segunda capa, con lo cual, además se está replicando el proceso de la misma forma: eliminando todos los datos y reconstruyéndolos. Es necesario mencionar lo propenso a errores que se encuentra todo lo anterior descrito, desde problemas estructurales hasta funcionales, entre ellos, los más destacados son: mal manejo de recursos de hardware causando que el servicio se congele, el desuso de codificación para almacenar datos, causando que metadatos se escriban en código, entre otros.
\section*{Problemática}
Mantener la consistencia de los datos de forma eficiente y segura constituye una tarea compleja. Lograr un sistema robusto, resistente a fallas.

\section*{Objetivo}
%Con el propósito de presentar una propuesta para solucionar la problemática expuesta anteriormente, se plantean los siguientes objetivos:

\subsubsection*{Objetivo General}

\begin{itemize}	
	\item Diseñar e implementar un sistema integrador de todos los usuarios de La Universidad de La Habana. 
\end{itemize}

\subsubsection*{Objetivos Específicos}
\begin{itemize}	
	\item Integrar todas las fuentes de usuarios de los servicios institucionales: Assets, Sigenu.
	\item Automatizar el acceso de generación de cuentas en tiempo real respaldado con sus sistemas de origen.
	\item Controlar el manejo de casos excepcionales y/o casos externos no aplicables a las fuentes de datos estándares.
	\item Implementar la gestión semi-automatizada del control de roles según metadatos descriptores del usuario.
	\item Generar los servicios de autenticación con compatibilidad en todas las tecnologías existentes y previstas en la institución.
	\item Implementar la gestión de controles de acceso para los diferentes servicios y la simplificación del proceso de añadir nuevos.
\end{itemize}