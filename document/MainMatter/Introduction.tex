\chapter*{Introducción}\label{chapter:introduction}
\addcontentsline{toc}{chapter}{Introducción}

%El Nodo Central es el responsable de todas las comunicaciones de la Universidad de La Habana. Presta servicios a todas las facultades, desde las que se encuentran en la Colina, hasta facultades externas como Economía, el Jardín Botánico y la Quinta de los Molinos. Servicios como Correo, acceso a Internet y proxy son  imprescindibles para garantizar el correcto funcionamiento de la institución. También es el encargado de proveer acceso a los estudiantes y trabajadores a todos los servicios digitales. Entre ellos se encuentra EVEA, el registro de notas, el sistema de contabilidad y el de recursos humanos. 

%Garantizar la seguridad de la información es uno de los mayores retos al que se enfrenta la institución actualmente.
%
%Una de las medidas de seguridad que se  utiliza es verificar  la identidad de los usuarios a través de un usuario y una contraseña  para acceder a los sistemas locales. Es importante que la autenticación del usuario sea correcta para que la aplicación permita un acceso correspondiente al rol de cada usuario.
%
%Todos los sistemas se han realizado de forma independiente a lo largo el proceso del digitalización de la Universidad. Todos tienen formas de acceso diferentes, lo cual significa que los usuarios para consumir los distintos servicios que utiliza diariamente tiene que autenticarse en cada de forma independiente. Esto crea dificultades tanto para los usuarios como para el Nodo Central. 

%Mecanismos de autenticación como el clásico usuario/contraseña son poco seguros o inadecuados, y más aún cuando los usuarios deben de acceder a cada aplicación por separado, ya que estas tienen su propio sistema de autenticación. Esto hace más engorroso el proceso de gestión de identidades y permisos a medida que aumenta la cantidad de usuarios en la red y debilitando la seguridad de acceso a las aplicaciones comprometiendo la confidencialidad de los datos.
%
%La necesidad de que el cliente repita el ingreso de la contraseña cada vez que accede a un servicio trae consigo varias dificultades: tiempo prolongado al tener que registrarse en varios sistemas para consultar información, la ubicación del servidor en una máquina remota implica inseguridad. Cuando la red crece mucho porque se encuentran una gran cantidad de usuarios conectados, puede volverse muy complejo el proceso de autenticación, es decir que es afectada la escalabilidad.
%
%Debido a esto se hace necesaria la gestión de la autenticación mediante la utilización de mecanismos que agilicen este proceso, sin que cause perdida de información, seguridad, confidencialidad o disponibilidad y que faciliten la gestión de grandes volúmenes de usuarios reduciendo el riesgo de caída de las aplicaciones y garantizando la gestión de identidades con el aumento de la seguridad, preferentemente con el uso de sistemas de identidad unificados


Hoy en día las personas utilizan muchos aplicaciones con distintas identidades. La identidad del usuario es considerada un conjunto de atributos, permanente o de larga vida, asociados a un usuario. Para obtener acceso a servicios muchas aplicaciones requieren que el usuario esté registrado y haya iniciado sesión. La mayoría de las veces los usuarios están registrados en distintos sitios o servicios con el mismo nombre de usuario y con la misma o similar contraseña, lo cual no es la mejor práctica para preservar la seguridad de las cuentas. Por todo ello el manejo de múltiples nombres de usuarios y contraseñas es una tarea molesta del Internet actual.

Teniendo en cuenta todos esos problemas de seguridad, cada vez más los administradores de sitios web deciden delegar esos servicios de administración y autenticación a terceros, entidades externas especializadas en ese tipo de actividad, que pueden alojar, almacenar, administrar y proteger los datos de los usuarios de un proveedor de servicios en particular. Además, ofrecen una interfaz de programación de aplicaciones (conocida también por la sigla API, en inglés, \textit{Application Programming Interface}), para proporcionar acceso a los datos del usuario a las aplicaciones externas. Estas soluciones permiten compartir los datos con más de un sistema web, permitiendo así el Inicio de Sesión Único (SSO) para aplicaciones de terceros.[\cite{kutera2016single}]

El Nodo Central de la Universidad de La Habana tiene entre sus responsabilidades dar las credenciales digitales a todos los usuarios de la Universidad. Los trabajadores y estudiantes de la institución registran sus datos personales en las bases de datos de recursos humanos y secretaria docente, respectivamente. Esa información es utilizada más adelante para que los usuarios puedan autenticarse en los distintos servicios de la institución, respaldado por sus sistemas de origen.

En este trabajo se presenta una solución a los problemas de autenticación que se ajusta a las necesidades y posibilidades de la Universidad de La Habana.  

\section*{Motivación}
Actualmente vivimos en un mundo donde la tecnología tiene un papel protagónico. La Universidad de La Habana en los últimos años ha estado inmersa en el proceso de transformación digital que lleva a cabo nuestro país, como continuidad de la estrategia de informatización de la sociedad cubana, que pretende integrar las tecnologías digitales a todos los ámbitos de la sociedad.

%En este sentido, nuestra sede universitaria ha avanzado en la digitalización de la información y los procesos y trabaja para facilitar el acceso a la red de todos sus estudiantes y trabajadores. A la par de estos avances, debe crecer también la seguridad de los sistemas para evitar brechas de información y asegurar la privacidad de los usuarios.

Garantizar un acceso seguro y sencillo a los recursos de la red es esencial para alcanzar este objetivo. Por consiguiente, se deben establecer mecanismos que cuenten con un alto nivel de seguridad y permitan identificar quién realmente está autorizado para acceder a los recursos del sistema.


%Nota Carmen: en que radica la mejora de la experiencia?????  para mi esto sobra a menos que digas en que mejora
%La creación de una plataforma central de autenticación resolvería muchos de estos problemas y mejoraría la experiencia de los usuarios. Estos solo tendrían una cuenta para acceder a todos los servicios que ofrece la Universidad, por lo que no tendrían que lidiar con formularios de inicio de sesión cada vez que vayan a acceder a un servicio.


\section*{Antecedentes}
%\textcolor{red} {Nota: Qué existe desde el punto de vista actual para autentificarte como usuario UH y sobre que esta soportado? qué problema trae ese soporte? existe en el mundo otros que pueden mejorar esas dificultades? Poner ejemplos. }

Actualmente todos los usuarios de la Universidad de La Habana se almacenan en dos Protocolos Ligeros de Acceso a Directorios (en inglés: \textit{Lightweight Directory Access Protocol}, también conocido por sus siglas como LDAP). En los dos directorios se guardan las cuentas de correo y los datos de sus usuarios. En uno ellos se registra la información de los estudiantes y en el otro la de los trabajadores. A partir de estos, todos los sitios web y aplicaciones de la universidad, individualmente, verifican la pertenencia del usuario a la institución.

Cada aplicación y servicio tiene un mecanismo de autenticación individual que depende de los dos mencionados directorios que contienen a los usuarios.

\section*{Problemática}
La Universidad brinda servicios como Wi-Fi, correo, proxy y EVEA (Entorno Virtual de Enseñanza Aprendizaje), imprescindibles para el funcionamiento de la institución. También brinda otros servicios más especializados como los sistemas de contabilidad, recursos humanos, inventarios, el registro de notas, entre otros. Todos ellos utilizan la autenticación de forma individual, lo cual significa que un usuario tiene diversas cuentas a pesar de pertenecer todas a la misma institución.

El sistema implementado actualmente es poco eficiente y requiere de intervención humana constante para corregir y/o restablecer el apropiado funcionamiento de los mecanismos de autenticación. Esto genera tiempos elevados de respuesta y dificulta el mencionado proceso de transformación digital que está siendo llevado a cabo por nuestra niversidad.

%\textcolor{red}{\section*{Hipotesis}}

\section*{Objetivo}
Con el propósito de presentar una propuesta para solucionar la problemática expuesta anteriormente, se plantean los siguientes objetivos:

\subsubsection*{Objetivo General}

\begin{itemize}	
	\item Diseñar e implementar un sistema de autenticación centralizada para todos los usuarios de La Universidad de La Habana. 
\end{itemize}

\subsubsection*{Objetivos Específicos}
\begin{itemize}	
	\item Generar los servicios de autenticación con compatibilidad con todas las tecnologías existentes y previstas en la institución.
	\item Implementar la gestión de control de acceso a todos los servicios ofrecidos por el Nodo Central de forma extensible a futuros servicios y fuentes de datos.
\end{itemize}

%\section*{\textcolor{red}{Estructura de la Tesis}}
%\textcolor{blue}{Aca realmente toca hablar de como has organizado la tesis. De cada capítulo que tema toca, sus secciones y siempre un poco de información al respecto}
