\chapter{Estado del Arte}\label{chapter:state-of-the-art}
En este capítulo se brindan las definiciones de herramientas utilizadas. También se realiza un estudio sobre el estado del arte de las mismas. Además, se brindan razones para incluir su utilización como parte de la solución propuesta.

\section{Keycloak}
Keycloak es un software de código abierto que permite el Single Sign-On o Inicio de Sesión Único (IdP) con Identity Management y Access Management para aplicaciones y servicios modernos. Esta herramienta facilita la protección de aplicaciones y servicios con poca o ninguna codificación. Un IdP permite que una aplicación (a menudo llamada Service Provider o SP) delegue su autenticación.

Este software está escrito en Java y es compatible de forma predeterminada con los protocolos de federación de identidad SAML v2 y OpenID Connect (OIDC) / OAuth2. Está bajo licencia de Apache y es compatible con Red Hat.

	\subsection{Características}
	Los usuarios se autentican en Keycloak en lugar de hacerlo en las aplicaciones. Esto significa que no es necesario que las aplicaciones tengan formularios de inicio de sesión, autentiquen a los usuarios o almacenen sus datos. Una vez entren en Keycloak, los usuarios no tendrán que iniciar sesión en las demás aplicaciones conectadas al software.
	
	Lo mismo sucede cuando un usuario cierra sesión. Keycloak ofrece cierre de sesión único, lo cual significa que los usuarios solo tienen que desconectarse en una de las aplicaciones para salir de su cuenta en el resto de las aplicaciones.
	
	Otra prestación de Keycloak son las federaciones de usuarios, que facilitan la compatibilidad con LDAP y otros servidores de directorios activos. También admite la implementación de servicios propios para usuarios guardados en otros tipos de almacenamientos como en bases de datos relacionales.
	
	Keycloak ofrece como herramienta una consola de administración de cuentas, a través de la cual los usuarios pueden administrar sus propias cuentas. Pueden actualizar su perfil, cambiar sus contraseñas y configurar la autenticación en dos pasos. También pueden administrar sus sesiones y visualizar el historial de su cuenta.
	
	También es una herramienta extensible porque permite la eliminación, adición y modificación de las bases de datos de usuarios, los métodos de autenticación y los protocolos. Está basada en protocolos estándares y soportan OpenID Connect, OAuth 2.0 y SAML.
	
	\subsection{Ventajas}
	Keycloak facilita añadir la autenticación y un servicio seguro a aplicaciones. Permite que los desarrolladores se centren en la funcionalidad empresarial al no tener que preocuparse por los aspectos de seguridad de la autenticación. También posibilita la unificación de los métodos de autenticación de distintas aplicaciones sin modificarlas.
	
\section{LDAP}
El Protocolo Ligero de Acceso a Directorios (en inglés: Lightweight Directory Access Protocol, también conocido por sus siglas de LDAP) es un conjunto de protocolos de licencia abierta que son utilizados para acceder a la información que está almacenada de forma centralizada en una red. Este protocolo se utiliza a nivel de aplicación para acceder a los servicios de directorio remoto.

LDAP está basado en estándares implementado sobre TCP/IP. Permite a los clientes interactuar directamente con los los servidores de los directorios: almacenar y consultar información, buscar datos filtrados, autenticar usuarios, entre otros.

Este protocolo es utilizado actualmente por muchos sistemas que apuestan por el software libre al utilizar distribuciones de Linux para ejercer las funciones propias de un directorio activo en el que se gestionarán las credenciales y permisos de los trabajadores y estaciones de trabajo en redes LAN corporativas en conexiones cliente/servidor.

Un directorio remoto es un conjunto de objetos que están organizados de forma jerárquica, tales como nombre claves direcciones, etc. Estos objetos estarán disponibles por una serie de cliente conectados mediante una red, normalmente interna o LAN, y proporcionarán las identidades y permisos para esos usuarios que los utilicen.

LDAP está basado en el protocolo X.500 para compartir directorios, y contiene esta información de forma jerarquizada y mediante categorías para proporcionarnos una estructura intuitiva desde el punto de vista de la gestión por parte de los administradores.

Estos directorios se utilizan generalmente para contener información virtual de usuarios, para que otros usuarios accedan y dispongan de información acerca de los contactos que están aquí almacenados. Además es capaz de comunicarse de forma remota con otros directorios LDAP situados en servidores que pueden estar en el otro lado del mundo para acceder a la información disponible. De esta forma se crea una base de datos de información descentralizada y completamente accesible.