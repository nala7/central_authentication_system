\lstset{language=Python}
\lstset{frame=lines}
\lstset{caption={Conexión de cliente a Keycloak}}
\lstset{label={lst:code_direct}}
\lstset{basicstyle=\footnotesize}
\begin{lstlisting}
from flask import Flask
from flask import Flask, render_template, request
from keycloak.keycloak_openid import KeycloakOpenID
from keycloak.exceptions import KeycloakAuthenticationError, KeycloakGetError

import json

app = Flask(__name__)

keycloak_open_id = KeycloakOpenID(server_url="http://localhost:8080/", 
	client_id="nodo", 
	realm_name="master", 
	client_secret_key="V35a9m9GjYpslkw11mXwEMHM3Ac8hmhD")
keycloak_open_id.well_know()

@app.route('/login', methods=['POST', 'GET'])
def login():
	error = None
	if request.method == 'POST':
		username = request.form['username']
		success, result = valid_login(username, request.form['password'])
		if success:
			return log_the_user_in(username, result["access_token"], result["refresh_token"])
		else:
			error = result["error_description"]
	# The code below is executed if the request method
	# was GET or the credentials were invalid
	return render_template('login.html', error=error)

def valid_login(username, password):
	# import pdb
	# pdb.set_trace()
	try:
		token = keycloak_open_id.token(username, password)
	except (KeycloakAuthenticationError, KeycloakGetError) as e:
		return False, json.loads(e.error_message)
	return True, token

def log_the_user_in(username, token, refresh_token):
	return render_template('success.html', username=username, token=token, refresh_token=refresh_token)
\end{lstlisting}
