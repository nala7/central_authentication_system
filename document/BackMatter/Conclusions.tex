\begin{conclusions}
    Durante el desarrollo de este trabajo se realizó un estudio del estado del arte acerca de Keycloak y las facilidades que brinda. Además, se estudiaron las mejores soluciones para la autorización y autenticación de usuarios.
    
    Se analizó el sistema que se utiliza actualmente en el Nodo Central para el almacenamiento de los usuarios y la autenticación de los mismos. Además, se procedió al diseño de un sistema más eficiente y con mejor experiencia de usuario. Esto contribuyó al cumplimiento del objetivo principal de este trabajo, la propuesta de una metodología que permite la autenticación de usuarios desde distintas aplicaciones de la institución de una forma eficiente y repetible.
    
    Como parte de la metodología se utilizan las herramientas LDAP para la administración de los datos de los usuarios y Keycloak para la autenticación y autorización. Se emplea el método Single Sign-On, la autenticación basada en \textit{tokens} y el protocolo OpenID Connect para garantizar un inicio de sesión sencillo y seguro.
    
    El sistema resultante es un mecanismo de autenticación que mejora respecto al que se encuentra en uso actualmente mediante el uso de herramientas más modernas y seguras. Dicho mecanismo permite a todos los clientes del Nodo Central autenticar y autorizar de forma fiable a sus usuarios mediante unas pocas líneas de código.
\end{conclusions}
