\begin{resumen}
	Las deficiencias del sistema de autenticación ofrecido por el Nodo Central de La Universidad de La Habana motivan la puesta en práctica de una nueva solución. Debido a las ventajas de Keycloak, se apuesta por esta herramienta como alternativa al sistema vigente. En consecuencia, se realiza un estudio del estado del arte acerca de Keycloak y el conjunto de protocolos y métodos que utiliza. Además, se presenta una solución a cómo configurar y hacer uso de esta herramienta.
	
	Se determinan las principales dificultades de los métodos de autenticación en uso en el Nodo Central y se procede a crear un nuevo sistema de autenticación central que permite integrar a todos los usuarios de la universidad en un único sistema. Este proceso culmina la propuesta de una metodología que permite a los clientes comunicarse con el nuevo sistema de forma sencilla y repetible. Como parte de la metodología se utilizan los protocolos \textit{OpenID Connect} y SAML para garantizar una comunicación rápida y segura. También se hace uso del método \textit{Single Sign-On} y la autenticación basada en tokens para facilitar la experiencia de los usuarios y LDAP para el almacenamiento de los datos.
	
	Al aplicar esta metodología se obtienen resultados satisfactorios: un sistema capaz de autenticar a todos los usuarios de la Universidad de La Habana a través de diversos clientes. Los resultados obtenidos validan la efectividad de la metodología desarrollada.

%	Authentication is crucial although if system which facilitates secure their networks by limiting access to protected resources such as networks, websites, network-based software, databases, and other computer systems or services to only authenticated users (or processes). In general, modern authentication protocols such as Security Assertion Markup Language 2.0 (SAML), WS-Fed, OAuth, and OpenID discourage apps from handling user credentials. The aim is to keep an app's authentication method and its functionality separate. Azure Active Directory (Azure AD) manages the login process to keep confidential data (such as passwords) out of the hands of websites and apps. This allows identity providers (IdP) like Azure AD to provide seamless single sign-on experiences, allow users to authenticate using factors other than passwords (phone, face, biometrics), and block or elevate authentication attempts if Azure AD detects, for example, that the user's account has been compromised or that the user is attempting to access an app from an untrusted location. The main goal of the work is Converting Visual Studio from ADAL to MSAL has allowed us to better support Conditional Access and Multi-factor Authentication and other new AAD features which benefit our customers. Visual Studio 2019 and the.NET Core SDK can be used to complete this work. The SAML request–response authentication workflow between these providers is checked to ensure that user login information is accurate and safe.

\end{resumen}

\begin{abstract}
	The inadequacy of the authentication system supplied by the Main Network Administration Center of the University of Havana motivates the implementation of a new solution. Given the advantages offered by Keycloak, this tool is chosen as an alternative to the current system. Consequently, a study of the state of the art about Keycloak and the set of protocols and methods it uses is carried out. In addition, a solution is presented on how to configure and make use of this tool.
	
	The main difficulties of the authentication mecanisms in use in the Main Network Administration Center are determined and a new central authentication system is created in order to integrate all university users into a single system. This process culminates in proposing a methodology that allows customers to communicate with the new system in a simple and repeatable way. As part of the methodology, OpenID Connect and SAML are used to guarantee fast and secure communication. In addition, Single Sign-On and token-based authentication are also used to facilitate the user experience.Finally, LDAP is utilized for data storage.
	
	Through this methodology, satisfactory results were obtained: a system capable of authenticating all users of the University of Havana through various clients. The results obtained validate the effectiveness of the developed methodology.
\end{abstract}