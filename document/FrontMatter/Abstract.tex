\begin{resumen}
	Resumen en español

%	Authentication is crucial although if system which facilitates secure their networks by limiting access to protected resources such as networks, websites, network-based software, databases, and other computer systems or services to only authenticated users (or processes). In general, modern authentication protocols such as Security Assertion Markup Language 2.0 (SAML), WS-Fed, OAuth, and OpenID discourage apps from handling user credentials. The aim is to keep an app's authentication method and its functionality separate. Azure Active Directory (Azure AD) manages the login process to keep confidential data (such as passwords) out of the hands of websites and apps. This allows identity providers (IdP) like Azure AD to provide seamless single sign-on experiences, allow users to authenticate using factors other than passwords (phone, face, biometrics), and block or elevate authentication attempts if Azure AD detects, for example, that the user's account has been compromised or that the user is attempting to access an app from an untrusted location. The main goal of the work is Converting Visual Studio from ADAL to MSAL has allowed us to better support Conditional Access and Multi-factor Authentication and other new AAD features which benefit our customers. Visual Studio 2019 and the.NET Core SDK can be used to complete this work. The SAML request–response authentication workflow between these providers is checked to ensure that user login information is accurate and safe.

\end{resumen}

\begin{abstract}
	Resumen en inglés
\end{abstract}