\begin{opinion}
La gestión y administración de sistemas es uno de los pilares que sustentan la ya creciente explosión del uso de las tecnologías, la informatización y las comunicaciones. Si bien en sus variantes más modernas de software como servicio e infraestructura como servicio se libera de la gestión al usuario final, todavía se hace necesario el manejo de infraestructura y servicios críticos, ya sea sobre hardware propio o de terceros.

En nuestro país, con la evolución del panorama tecnológico de los recientes años, se ha incrementado la demanda activa de personal cuyas habilidades y conocimientos puedan conducir el desarrollo tecnológico que la nación se ha propuesto alcanzar. Temas como la gestión de recursos humanos e inventario, la ciberseguridad y monitorización, la autenticación y autorización, gestión de dominio y la retroalimentación de cara a los usuarios finales se han convertido en las bases sobre las que se sustentan la sociedad informatizada que queremos construir. 

Dentro de estos objetivos, en particular, sobre la gestión de la autenticación central para la Universidad de La Habana se desarrolla la tesis de la estudiante Nadia González Fernández. La estudiante propone una arquitectura de solución que integra las disímiles fuentes de datos que existen hoy en nuestra casa de altos estudios, dando paso a una solución extensible, simple y de gestión centralizada al ecosistema, heterogéneo, de autenticación que se maneja hoy en nuestra infraestructura. 

Nadia, en este documento de tesis, desarrolla una solución de inicio de sesión único sobre una plataforma de autenticación, en particular sobre Keycloak que se incorpora sobre las dos bases de datos LDAP que existen hoy en día en la Universidad. Dicha solución extiende el panorama de la autenticación en la Universidad con dos modernos protocolos, OpenId Connect y SAML sobre los cuales se proyecta sustentar toda la autenticación que se maneja actualmente. Finalmente, valida su propuesta e implementación mediante una prueba de carga realizada mediante Locust y una prueba de validación de los protocolos propuestos en Flask.

La estudiante, durante el desarrollo de esta tesis, ha honrado de una forma u otra todos los conocimientos impartidos que se esperan de un graduado de nuestra institución. Ha realizado un estudio del estado del arte, consultando tanto documentación técnica como estudios científicos. Ha propuesto y ha llevado a cabo una arquitectura de solución, extensible, moderna y creativa al problema de la autenticación segregada de la Universidad. Ha mostrado, también, las habilidades de comunicación necesarias para transmitir sus resultados. Finalmente, ha hecho gala de una fuerza de voluntad admirable que le ha permitido sobreponerse a todos los problemas presentados en la realización de esta tesis. 

Por estos motivos, considero que Nadia González Fernández ha demostrado con creces haber adquirido las habilidades que la avalan como una excelente Científica de la Computación y estimo razonable solicitar al tribunal que se le otorgue la máxima calificación.

Solo me queda desearle el mayor de los éxitos en su vida profesional. Que siga honrando a nuestra institución como bien ha hecho en este documento y que siga cosechando los frutos de su esfuerzo donde sea que los vientos de la vida le terminen llevando.


\begingroup
\centering
\wildcard{Lic. Robert Marti Cedeño}
\par
\endgroup


\end{opinion}